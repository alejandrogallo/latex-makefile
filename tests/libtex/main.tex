%   preamble {{{1  %
%%%%%%%%%%%%%%%%%%%%

%\documentclass[notes=show]{beamer}
%\documentclass[notes=hide]{beamer}
\documentclass[notes=hide,handout]{beamer}

\usepackage[utf8]{inputenc}
\usepackage[english]{babel}
\usepackage[]{tikz}

\newcommand{\ket}[1]{\left| #1 \right>} % for Dirac bras
\newcommand{\bra}[1]{\left< #1 \right|} % for Dirac kets
\newcommand{\braket}[2]{\left< #1 \vphantom{#2} \right|\!\!
  \left. #2 \vphantom{#1} \right>} % for Dirac brackets

%  beamer theme {{{1  %
%%%%%%%%%%%%%%%%%%%%%%%

\usetheme{metropolis}           % Use metropolis theme
\metroset{background=light}
\setbeamertemplate{frame footer}{%
\includegraphics[width=1cm, keepaspectratio]{images/max_planck.png}
} %Metropolis defined
\setbeamercolor{background canvas}{bg=white}



\title[Something \ldots]{%
  \vspace{1.5cm}
  Some random presentation
}
\date{March 15, 2017}


\author{Someone}
\institute{%
  Max-Planck Institute for Makefile research\\
  Stuttgart, Germany\\
}


\begin{document}


%  title {{{1  %
%%%%%%%%%%%%%%%%

\maketitle





\begin{frame}{Basic level overview} %{{{1
  \begin{center}
    \includegraphics[height=0.7\textheight]{images/basic_level_definition.pdf}
  \end{center}
\end{frame}

\begin{frame}{Basic level overview: $ \mathsf{NV}^{-} $ Ground State} %{{{1
  \begin{center}
    \includegraphics[height=0.7\textheight]{images/basic_level_definition_with_spin.pdf}
  \end{center}
\end{frame}

\begin{frame}{Triplets overview: $ \mathsf{NV}^{-} $ } %{{{1
  \begin{center}
    \begin{columns}
      \begin{column}{0.5\textwidth}
        \centering
        \includegraphics[height=.8\textheight]{images/basic_levels_triplets.pdf}
      \end{column}
      \begin{column}{0.5\textwidth}
        \centering
        \includegraphics[height=1\textheight]{images/basic_triplet_configuration.pdf}
      \end{column}
    \end{columns}
  \end{center}

  \note{%

    If you have never heard of this defect before, you might be wondering how
    exactly the electrons in the solid body come into play.  In the one
    particle picture of the problem it turns out that we can characterize very
    well these levels by the occupation of the valence states in the body.

    In this picture we can associate to every state with an occupation of the
    topmost states. From this picture we can see that the total spin is one and
    the excitation happens through the promotion of one electron in the level $
    a_{1} $ into one of the degenerated $ e $ levels. This without incurring in
    a spin flip process.

    This we should also find in our calculations.

  }

\end{frame}

\begin{frame}{Main level overview: $ \mathsf{NV}^{-} $ } %{{{1
  hello

  \note{%

    To explain fluorescence experiments it is essential to have a picture of
    the electronic and vibronic states of a defected diamond sample.
    Theoretical models using a \textit{Linear combination of atomic orbitals}
    (LCAO) give rise to a good prediction of the Spin multiplicity and triplet
    energy levels of the negative NV center.

    From the interplay of theory and experiments arises this electronic level
    scheme. On the one hand we encounter two stable triplet levels, where one
    of them is the ground state of the defect. Between them lie two meta-stable
    singlet states, which will play an essential role for applications of the
    defect levels.

  }

\end{frame}












\plain{Thank you!}







\end{document}


% vim: spell fdm=marker :
